\chapter{Two View Reprojection}

\section{Overview}
The goal with the 2-view reprojection is to take the pairs of matched points from the SIFT steps and
move them from $\mathbb{R}^{2}$ into $\mathbb{R}^{3}$ so that equations of lines can be generated from the
focal point of the camera into each matched point. Then, we want to find the minimum distance between those
lines and choose the midpoint of that line segment at our reprojected point.

\section{Getting Equations of Lines}

\begin{center}
\tdplotsetmaincoords{-60}{-35}
\begin{tikzpicture}
  [
    tdplot_main_coords,
    >=Stealth,
    my dashed/.style={dashed, thick, ->, shorten >=-15pt, shorten <=-15pt, every node/.append style={font=\footnotesize}},
    my box/.style={thin, gray!70},
    my blue/.style={blue, line cap=round, -{Triangle[width=3*#1]}, line width=#1, shorten >=#1*1.75pt, every node/.append style={fill, circle, inner sep=0pt, minimum size=#1*3.5pt, anchor=center, outer sep=0pt}},
    my label/.append style={midway, font=\scriptsize},
    my vectors/.style={green!50!black, {Stealth[scale=.75]}-{Stealth[scale=.75]}},
    my red/.style={thick, red, line cap=round},
    my grey/.style={gray!70},
    description/.style={draw=gray!70, thick, line cap=round, every node/.style={align=center, font=\scriptsize\sffamily, anchor=north}},
  ]
  \draw [my grey] (0,4,0) -- (0,7,0) (-2,7,0) -- (2,7,0);
  \coordinate (o) at (0,0,0);

  \path [draw=gray!70, text=gray, fill=gray!20, opacity=0.8, text opacity=1] (-1.5,4,1.75) coordinate (a) -- ++(0,0,-3.5) coordinate (b) -- ++(3,0,0) coordinate (c) -- ++(0,0,3.5) coordinate (d) -- cycle node [pos=.95, above, sloped, anchor=south west] {$z=f$} ;

  \draw [my grey] (-2,0,0) -- (2,0,0) (0,0,0) -- (0,4,0) (0,0,0) -- (0,0,2);
  \draw [thick, ->, every node/.style={font=\footnotesize, inner sep=0pt}] (o) node [anchor=north west] {$C$} (o) edge node [pos=1, anchor=north east] {$C_z$} ++(0,1,0) edge node [pos=1, anchor=north] {$C_y$} ++(0,0,1) -- ++(1,0,0) node [anchor=north west] {$C_x$};
  \draw [my box] (o) ++(0,4,-.5) coordinate (p1) -- ++(1,0,0) coordinate (p2) -- ++(0,0,-1.25) coordinate (p3);
  \foreach \i in {0,1,...,4} \draw [my box] (p1) ++(\i*.25,0,0) -- ++(0,0,-.25);
  \foreach \i in {0,1,...,5} \draw [my box] (p2) ++(0,0,-\i*.25) -- ++(-.25,0,0);

  \draw [my box] (p1) ++(0,0,-.25) -- ++(.75,0,0) -- ++(0,0,-1);
  \draw [my dashed, cyan] ($(b)!1/2!(c)$) -- ($(d)!1/2!(a)$) node [below=15pt, anchor=north] {$y - \frac{res}{2}$};
  \draw [my dashed, cyan] ($(b)!1/2!(a)$) -- ($(d)!1/2!(c)$) node [above right=17pt, anchor=north west] {$x - \frac{res}{2}$};

  \draw [my dashed, green!50!black, <->] (a) node [below=15pt, anchor=north] {$x$} -- (b) -- (c) node [above right=17pt, anchor=north west] {$y$};
  \path [green!50!black, every node/.style={font=\scriptsize, inner sep=0pt}] (p2) node [above right, anchor=south west] {$(x',y')$};

  \path (p2) ++(-.125,0,0) coordinate (q2) ++(0,0,-.125) coordinate (r2);
  \draw ($(0,4,0)+($(q2)-(p1)$)$) coordinate (s2) -- (r2) node (d1) {};
  \scoped[on background layer]{\draw ($($1.75*($(s2)-(0,4,0)$)$)+(0,7,0)$) -- ++($1.75*($(r2)-(s2)$)$) node (d2) {};}
  \draw [my red] (o) node[above] {$\vec{v}$} -- (d1.center);

  \scoped[on background layer]{\draw [my red] (d1.center) -- (d2.center);}

\end{tikzpicture}
\end{center}

The goal here is to generate a parametric equation of a line given camera position and orientation coordinates $C$,
the camera focal length $f$, and the position of a coordinate in $\mathbb{R}^{2}$ on the image plane. We wish to
generate vector $v$ that can be used to make the parametric equation.
The 2-view reprojection takes the matched points between 2 images and places them into $\mathbb{R}^{3}$.
To place each set of points into $\mathbb{R}^{3}$ some trigonometry and and matrix transformations need to take place.
The first step to moving a keypoint into $\mathbb{R}^{3}$ is to place it onto a plane in $\mathbb{R}^{2}$.
the coordinates $(x', y')$ in $\mathbb{R}^{2}$ require the size of a pixel $dpix$, the location of the keypoint $(x, y)$,
and the resolution of the image $(xres, yres)$ to yield:

\[
x' = dpix(x - \frac{xres}{2})\hspace{1cm} y' = dpix(y - \frac{yres}{2})
\label{test}
\]

This is repeated for the other matching keypoint. The coordinate $(x', y', z')$ in $\mathbb{R}^{3}$ of
the keypoint $(x', y')$ in $\mathbb{R}^{2}$ is given by three rotation matrices and one translation matrix.
First we treat $(x', y')$ in $\mathbb{R}^{2}$ as a homogenous vector in $\mathbb{R}^{3}$ to yield $(x', y', 1)$.
Given a unit vector representing the camera, in our case the spacecraft’s
camera’s, orientation $(r_x,r_y,r_z)$ we find the angle to rotate in each axis $(\theta_x,\theta_y,\theta_z)$.
in our simple case we find the angle in the $xy$ plane with:

\[
\theta_z = \cos^{-1} \frac{ (\begin{bmatrix}1 & 0 & 0\end{bmatrix} \cdot \begin{bmatrix}r_x & r_y & r_z \end{bmatrix}) }{ \sqrt{ \begin{bmatrix}r_x & r_y & r_z\end{bmatrix} \cdot \begin{bmatrix}r_x & r_y & r_z\end{bmatrix} } }
\]

Future software will generate rotations for all planes in an identical way.
Now, given a rotation in each plane $(\theta_x,\theta_y,\theta_z)$ we calculate the homogeneous
coordinate $(r_x,r_y,r_z, 1)$ in $\mathbb{R}^{4}$ using linear transformations.
The values $(T_x,T_y,T_z)$ represent a translation in $\mathbb{R}^{3}$ and use camera
position coordinates $(C_x,C_y,C_z)$, the camera unit vectors
representing orientation $(u_x,u_y,u_z)$, and focal length $f$:
%=========
\[
\begin{bmatrix}
  1 & 0 & 0 & 0\\
  0 & \cos(\theta_x) & -\sin(\theta_x) & 0\\
  0 & \sin(\theta_x) & \cos(\theta_x) & 0\\
  0 & 0 & 0 & 1
\end{bmatrix}
\begin{bmatrix}
  \cos(\theta_y) & 0 & \sin(\theta_y) & 0\\
  0 & 1 & 0 & 0\\
  -\sin(\theta_y) & 0 & \cos(\theta_y) & 0\\
  0 & 0 & 0 & 1
\end{bmatrix}
\begin{bmatrix}
  \cos(\theta_z) & -\sin(\theta_z) & 0 & 0\\
  0 & \cos(\theta_x) & -\sin(\theta_x) & 0\\
  0 & 0 & 1 & 0\\
  0 & 0 & 0 & 1
\end{bmatrix}
\begin{bmatrix}
  x\\
  y\\
  z\\
  1
\end{bmatrix}
=
\begin{bmatrix}
  x_r\\
  y_r\\
  z_r\\
  1
\end{bmatrix}
\]
%=========
\[
\begin{bmatrix}
  C_x - (x_r + f * u_x)\\
  C_y - (y_r + f * u_y)\\
  C_z - (z_r + f * u_z)\\
  1
\end{bmatrix}
=
\begin{bmatrix}
  T_x\\
  T_y\\
  T_z\\
  1
\end{bmatrix}
\]
%=========
\[
\begin{bmatrix}
  1 & 0 & 0 & T_x\\
  0 & 1 & 0 & T_y\\
  0 & 0 & 1 & T_z\\
  0 & 0 & 0 & 1
\end{bmatrix}
\begin{bmatrix}
  x_r\\
  y_r\\
  z_r\\
  1
\end{bmatrix}
=
\begin{bmatrix}
  x'_r\\
  y'_r\\
  z'_r\\
  1
\end{bmatrix}
\]

The above process should happen for both points that have been matched.
This should result in 2 homogeneous points that we will call $\begin{bmatrix}x_0 & y_0 & z_0 & 1 \end{bmatrix}^T$
and $\begin{bmatrix}x_1 & y_1 & z_1 & 1 \end{bmatrix}^T$. Each point has a corresponding camera vector,
which is already known thanks to the camera coordinates, $\begin{bmatrix}C_{x0} & C_{y0} & C_{z0} & 1 \end{bmatrix}^T$
and $\begin{bmatrix}C_{x1} & C_{y1} & C_{z1} & 1 \end{bmatrix}^T$. From this we can make parametic lines $L_0$ and $L_1$
with the parametic variables $t_0$ and $t_1$:

\[
L_0 =
\begin{bmatrix}
  x_0 - C_{x0}\\
  y_0 - C_{y0}\\
  z_0 - C_{z0}
\end{bmatrix}
\begin{bmatrix}
  t_0\\
  t_0\\
  t_0
\end{bmatrix}
+
\begin{bmatrix}
  C_{x0}\\
  C_{y0}\\
  C_{z0}
\end{bmatrix}
\]

\[
L_1 =
\begin{bmatrix}
  x_1 - C_{x1}\\
  y_1 - C_{y1}\\
  z_1 - C_{z1}
\end{bmatrix}
\begin{bmatrix}
  t_1\\
  t_1\\
  t_1
\end{bmatrix}
+
\begin{bmatrix}
  C_{x1}\\
  C_{y1}\\
  C_{z1}
\end{bmatrix}
\]

% the other alg

The Host functions are simple and will not be addressed here, refer to standard CUDA memory management.
For the Kernel, we must insure that we have a few global variables about the camera data/parameters
availible to the GPU. These variables are the focal length $foc$, the feild of view $fov$, and the
resolution of the imager $res$. This assumes that the cameras are in the $xy$ plane, which can only
be assumed for this special 2-view case. It may be helpful to read the section on file formats, which
is after the main introduction.

\begin{algorithm}
\caption{Device Kernel}\label{Line Generation Device}
\begin{algorithmic}[1]
  %inside alg
  \State Let: $dpix = foc * \tan{\frac{fov/2}{res/2}}$
  \State Let: $i = blockIdx.x * blockDim.x + threadIdx.x$
  \Procedure{GenerateLines(R2points, R3cameras)}{}
    \State $C_0[6] = (R3cameras[0],R3cameras[1],R3cameras[2],R3cameras[3],R3cameras[4],R3cameras[5])$
    \State $C_1[6] = (R3cameras[6],R3cameras[7],R3cameras[8],R3cameras[9],R3cameras[10],R3cameras[11])$
    \State $x_0 = (R2points[i] - res/2.0)$
    \State $y_0 = (-R2points[i+1]) + res/2.0)$
    \State $x_1 = (R2points[i+2] - res/2.0)$
    \State $y_1 = (-R2points[i+3]) + res/2.0)$
    \State $kp_0 = [x_0,y_0,0.0]$
    \State $kp_1 = [x_1,y_1,0.0]$
    \State $\theta_{x0} = \cos^{-1}(\frac{kp0 \cdot \begin{bmatrix}1 & 0 & 0\end{bmatrix}^{T}}{\sqrt{\|kp_0\|}})$
    \State $\theta_{x1} = \cos^{-1}(\frac{kp1 \cdot \begin{bmatrix}1 & 0 & 0\end{bmatrix}^{T}}{\sqrt{\|kp_1\|}})$
    \State $kp_0 = kp_0 \begin{bmatrix}1 & 0 & 0\\ 0 & \cos(\frac{\pi}{2}) & -\sin(\frac{\pi}{2})\\ 0 & \sin(\frac{\pi}{2}) & \cos(\frac{\pi}{2})\\ \end{bmatrix}    \begin{bmatrix}\cos(\theta_{x0} + \frac{\pi}{2}) & -\sin(\theta_{x0} + \frac{\pi}{2}) & 0\\ \sin(\theta_{x0} + \frac{\pi}{2}) & \cos(\theta_{x0} + \frac{\pi}{2}) & 0\\ 0 & 0 & 1\\ \end{bmatrix}$
    \State $kp_1 = kp_1 \begin{bmatrix}1 & 0 & 0\\ 0 & \cos(\frac{\pi}{2}) & -\sin(\frac{\pi}{2})\\ 0 & \sin(\frac{\pi}{2}) & \cos(\frac{\pi}{2})\\ \end{bmatrix}    \begin{bmatrix}\cos(\theta_{x1} + \frac{\pi}{2}) & -\sin(\theta_{x1} + \frac{\pi}{2}) & 0\\ \sin(\theta_{x1} + \frac{\pi}{2}) & \cos(\theta_{x1} + \frac{\pi}{2}) & 0\\ 0 & 0 & 1\\ \end{bmatrix}$
    \State $kp_0[0] = C_0[0] - (kp_0[0] (C_0[3] * foc))$
    \State $kp_0[1] = C_0[1] - (kp_0[1] (C_0[4] * foc))$
    \Comment kp0 now represents the keypoint location in $\mathbb{R}^{3}$
    \State $kp_1[0] = C_1[0] - (kp_1[0] (C_1[3] * foc))$
    \State $kp_1[1] = C_1[1] - (kp_1[1] (C_1[4] * foc))$
    \Comment kp1 now represents the keypoint location in $\mathbb{R}^{3}$
    \State $v_0 = kp_0 - C_0$
    \Comment line $L_0$'s vector
    \State $v_1 = kp_1 - C_1$
    \Comment line $L_1$'s vector
  \EndProcedure
\end{algorithmic}
\end{algorithm}

In the future, only the first $3$ components, the position coordinates, of $C_0$ and $C_1$ are used.
Vectors $v_0$ and $v_1$ are used to parateterizeds lines.

\newpage
\section{Minimum Distance Between Skew Lines}
Now that we have lines $L_0$ and $L_1$, the challenge is to find the points $s_0$ and $s_1$ of
closest approach. First, we must test the assumption that our lines are skew, meaining they
are not parrallel and do not intersect. To start to think of this we take the forms of $L_0$ and $L_1$
and simplify them by thinking of them as parametic vectors where $C_0$ and $C_1$, represent the camera
position vectors and $v_0$ and $v_1$ represent the vector was previously calculated from the subtraction
of match coordinates with the camera vector. We make the simple equations:

\[
L_0 = v_0 t_0 + C_0 \hspace{1cm} L_1 = v_1 t_1 + C_1
\]

To make sure that the lines are not parallel, which is unlikely, we must verify that their cross product
is not zero. if $v_0 \times v_1 = 0$ then we have a degenerate case with infinitely many solutions. As long
as we know this is not the case we can proceed. We know that the cross product of the two vectors
$c = v_0 \times v_1$ is perpendicular to the lines $L_0$ and $L_1$. We know that the plane $P$, formed by the
translation of $L1$ along $c$, contains $C_1$. We also know
that the point $C_1$ is perpendicular to the vector $n_0 = v_1 \times (v_0 \times v_1)$. Thus, the intersection
of $L_0$ with $P$ is also the point, $s_0$, that is nearest to $L_1$, given by the equation:

\[
s_0 = C_0 + \frac{(C_1 - C_0) \cdot n_0}{v_0 \cdot n_0} \cdot v_0
\]

This also holds for the second line $L_1$, the point $s_1$, and vector $n_1 = v_0 \times (v_1 \times v_0)$.
with the equation:

\[
s_1 = C_1 + \frac{(C_0 - C_1) \cdot n_1}{v_1 \cdot n_1} \cdot v_1
\]

Now, given to points that represent the closest points of approach, we simply find the midpoint $m$:

\[
m = \begin{bmatrix}
  (s_0[x] + s_1[x])/2\\
  (s_0[y] + s_1[y])/2\\
  (s_0[z] + s_1[z])/2
\end{bmatrix}
\]

\section{Least Square Approximation for Two Lines}
