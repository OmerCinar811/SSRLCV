\chapter{Bundle Adjustment}

\section{Introduction}
Bundle adjustment is defined as the problem of refining a visual reconstruction of a scene geometry
to produce jointly optimal 3D point cloud representation of the scene and viewing parameter
(camera pose and calibration) estimates.

\section{Notation}
Here we list some notation used in this paper. \\

\begin{tabular}{p{1.5cm}p{15cm}p{1cm}}
  $I$ & $\{I_1, ..., I_N\}$. Input set of N images \\
  $C_i$ & camera center location corresponding to $I_i$ (world coordinates) \\
  $c_{foc}$ & camera focal length \\
  $c_{fov}$ & camera field of view \\
  $K$ & intrinsic camera matrix \\
  $R_i$ & $i$-th camera rotation matrix \\
  $M_i$ & $i$-th camera projection matrix \\
  $PC$ & $\{P_1, ..., P_L\}$. Point cloud set \\
  $P_j$ & $\{x_j, y_j, z_j\}$. $i$-th computed 3D feature point (world coordinates) \\
  $p_{(i,j)}$ & observed 2D image point on image $i$ corresponding to $P_j$ (pixel coordinates) \\
  $s$ & state vector \\
  $\delta$ & step vector \\
  $f(s)$ & cost/error function \\
\end{tabular}

\section{Problem Formulation}
We are given a set of $N$ images along with the feature points for each image. As a prerequisite,
a set of 3D points, $PC$, are estimated for each pixel point match set. It is convenient to define a
``track'' for each $P_j \in PC$ as
$$ T_j = \{P_j, Q(P_j)\} $$
where $Q(P_j) = \{p_{(1,j)}, p_{(2,j)}, ..., p{(L,j)}\}$ is the set of observed image points that
were previously used to predict $P_j$, and where $p_{(i,j)} = null$ whenever no point on $I_i$
contributed to estimation of $P_j$ (i.e., when $P_j$ is not visible on $I_i$). \newline
[[ reword this after writing earlier parts of pipeline to refer to ]] \newline
Then as we previously defined, for a particular image point $p_{(i, j)}$, the projection error,
or distance between the observed location and the projected image point location
$\bar{p}_{(i, j)} = M_i P_j$, is denoted
$$d(p_{(i,j)}, \bar{p}_{(i,j)})$$
This distance function depends on all entries in the projection matrix $M_i$, namely $R_i$, $C_i$, 
$c_{foc}$, and $c_{fov}$, as well as the 3D coordinates of each feature point
$P_j = (x_j, y_j, z_j)$. Now we organize all of these variables into one state vector $s$:
$$ s = \{\ \bigcup_{i}R_i, \ \bigcup_{i}C_i, \ c_{foc}, \ c_{fov}, \ \bigcup_{j}P_j \ \} $$
with $i = 1,2, \dots, N$ and $j = 1,2, \dots, L$. Then our goal is to globally minimize the amount
of projection error with respect to $s$. We define the cost function
\begin{equation}
  f(s) = \sum_{i=i}^{N} \sum_{j=1}^{L} d( p_{(i,j)}, \ \bar{p}_{(i,j)} ) =
  \sum_{i=i}^{N} \sum_{j=1}^{L} d( p_{(i,j)}, \ M_i P_j )
\end{equation}
The goal of bundle adjustment then is to compute the optimal state vector parameters to minimize
the cost function:
\begin{equation}
  \min_{s} f(s)
\end{equation}

\section{BA Algorithm}
This section explains the bundle adjustment algorithm in detail. First we explain the mathematical
solution to the problem expressed in section 3. After that we describe the methods used to reduce
computational cost so that the algorithm becomes feasible. 

\subsection{Numerical Optimization Estimate}
We need to minimize the cost function $f(s)$ over parameters $s$ starting from the given initial
estimate. Real valued cost functions are too complicated to minimize in closed form, and the
parameter space is non linear, so instead we approximate $f(s)$ by a local linear model - in our
case a quadratic Taylor series expansion of $f$ at the current state $s$. For a small step
$\delta$,
\begin{equation}
  f(s + \delta) \approx f(s) + g(s)^T \delta + \frac{1}{2} \delta^T H(s) \delta
\end{equation}
where $g \equiv \frac{df}{ds}(s)$ and $H \equiv \frac{d^2f}{ds^2}(s)$ are the gradient vector
and Hessian matrix of f, respectively. Assuming that H is positive definite, the local model is
a simple quadratic with a unique global minimum, which can be found explicitly. To do this, we
use the Gauss-Newton method to solve for step $\delta$ in the following linear system.
\begin{equation}
  H(s) \delta = -g(s)
\end{equation}

\subsection{Schur Complement}
Note that calculating the second derivatives $H$ and then inverting the matrix to solve (4) by
$$ \delta = -H^{-1} g $$
is extremely computationally expensive for complex cost functions such as this $f$, so bundle
adjustment methods typically estimate the hessian using clever techniques such as exploiting
the sparse nature of $H$.

\section{Pseudocode}

\section{Appendix}

\end{document}
